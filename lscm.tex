\documentclass[12pt]{article}
\author{a.shoobovych}
\usepackage{mathtools}
\usepackage{listings}
\usepackage{color}
\usepackage{picture}

\begin{document}

	\section{LSCM}
		Let's perform LCSM (Least Square Conformal Maps) algorithm!
		\\ \\
		First of all, here's the algorithm:
		
		\begin{enumerate}
			\item split entire model to charts
			\item unwrap charts
			\item pack charts to one map
		\end{enumerate}
			
		More detailed algorithm description:
		
		\subsection{split entire model to charts}
			\begin{enumerate}
				\item detect features defining chart boundaries
				\item grow features (thin them)
				\item expand charts (create them)
				\item validate charts (optional; overlaps)
			\end{enumerate}
			
			\subsubsection{Feature Detection}
				\setlength{\unitlength}{1cm}
				\begin{picture}(11,6)
					\put(0,0){\vector(0,1){5}}
					\put(0,0){\vector(1,0){10}}
					\put(0,0){\vector(4,3){4.5}}
				\end{picture}
		
		\subsection{unwrap charts (LSCM itself)}
			\begin{center}
				$X = (A^{T} A)^{-1} A^{T} b$.
			\end{center}
			
			\emph{x} is a vector of $(u, v)$ pairs. It is our goal!
			\\
			
			\begin{displaymath}
				A = \begin{pmatrix}
					M^{1}_{f} & -M^{2}_{f} \\
					M^{2}_{f} & M^{1}{f} \\
				\end{pmatrix}
			\end{displaymath}
				
			$M$ is a matrix (see below) of complex numbers. It is split into two parts:
			\\
			
			\begin{displaymath}
				M \Leftrightarrow 
				\begin{pmatrix}
					M_{f} & M_{p} \cr
				\end{pmatrix}
				\Leftrightarrow 
				\begin{pmatrix}
					 M_{n' \times (n - p)} & M_{n' \times p} \\
				\end{pmatrix}
			\end{displaymath}
			
			\begin{displaymath}
				p = 2 ;\qquad 
				n' = triangleCount ;\qquad 
				n = vertexCount ;\qquad 
			\end{displaymath}
			
			\begin{displaymath}
				M \Leftrightarrow 
				\begin{pmatrix}
					M_{n' \times (n - 2)} & M_{n' \times 2} \\
				\end{pmatrix}
			\end{displaymath}
	
			Up-indices $^{1}$ and $^{2}$ stand for \emph{real} and \emph{imaginary} parts of complex numbers, respectively.		
			\\
			
			\begin{displaymath}
				b = 
				\begin{pmatrix}
					M^{1}_{p} & -M^{2}_{p} \\ 
					M^{2}_{p} & M^{1}_{p} \\
				\end{pmatrix}
				\begin{pmatrix}
					U^{1}_{p} \\
					U^{2}_{p} \\ 
				\end{pmatrix}
			\end{displaymath}
			
			\begin{displaymath}
				M = (m_{i,j})
			\end{displaymath}
			
			\begin{displaymath}
				m_{i,j} = 
				\begin{cases}
					\frac{W_{j, T_{i}}}{\sqrt{d_{T_{i}}}} & \text{if Vert[j]} \in \triangle T_{i} \\
					0 & \text{otherwise} \\
				\end{cases}
			\end{displaymath}
			
			Here, $d_{T_{i}}$ is the doubled square of $\triangle T_{i}$: 
			
			\begin{displaymath}
				d_{T_{i}} = 2S_{\triangle T_{i}}
				\begin{cases} 
					W_{j,1} = (x_{3} - x_{2}) + i(y_{3} - y_{2}) \\ 
					W_{j,2} = (x_{1} - x_{3}) + i(y_{1} - y_{3}) \\ 
					W_{j,3} = (x_{2} - x_{1}) + i(y_{2} - y_{1}) \\ 
				\end{cases}
			\end{displaymath}
			
			So, just calculate the right index for i ($W_{j,i}$, in the first formula; here $i = \sqrt{-1}$) and use either $W_{j,1}$, $W_{j,2}$ or $W_{j,3}$
			\\
			
			$U_{p}$ is a vector of $u + iv$ complex numbers determining UV-coordinates for \emph{pinned} vertices (sub-index of \emph{p})
			\\
	
	\pagebreak
			
	\section{Implementation}
		\subsection{Feature detection}
		
			some text
			
			\definecolor{dkgreen}{rgb}{0,0.6,0}
			\definecolor{gray}{rgb}{0.5,0.5,0.5}
			\definecolor{mauve}{rgb}{0.58,0,0.82}
			
			\lstset{
				basicstyle=\footnotesize,           % the size of the fonts that are used for the code
				numbers=left,                   % where to put the line-numbers
				numberstyle=\footnotesize,          % the size of the fonts that are used for the line-numbers
				%
				numbersep=5pt,                  % how far the line-numbers are from the code
				backgroundcolor=\color{white},      % choose the background color. You must add \usepackage{color}
				showspaces=false,               % show spaces adding particular underscores
				showstringspaces=true,         % underline spaces within strings
				showtabs=false,                 % show tabs within strings adding particular underscores
				frame=single,                   % adds a frame around the code
				tabsize=2,                      % sets default tabsize to 2 spaces
				captionpos=b,                   % sets the caption-position to bottom
				breaklines=true,                % sets automatic line breaking
				breakatwhitespace=false,        % sets if automatic breaks should only happen at whitespace
				title=\lstname,                   % show the filename of files included with \lstinputlisting;
								% also try caption instead of title
				numberstyle=\tiny\color{gray},        % line number style
				keywordstyle=\color{blue},          % keyword style
				commentstyle=\color{dkgreen},       % comment style
				stringstyle=\color{mauve},         % string literal style
				escapeinside={\%*}{*)},            % if you want to add a comment within your code
				morekeywords={*,...}               % if you want to add more keywords to the set
			}
		
			\begin{lstlisting}[frame=single,language={C++},caption={moo}]
				#include <irrlicht.h>
				
				using namespace irr;
				
				int main()
				{
					printf("hello, Wooooooooooooooooooooooooooooooooooooooooooorld!"); moo; foo; bar; baz;
					
					return 0;
				}
			\end{lstlisting}
			
\end{document}